\documentclass{jsarticle}
\usepackage{img-sys}
\usepackage[dvipdfmx]{graphicx}

\begin{document}
\title{画像システム特論発表用レジュメ}
\author{画像太郎}

\setlength{\baselineskip}{4.4mm}
\maketitle
\thispagestyle{empty}
\pagestyle{empty}

\section{概要}

「1.自分で実装してみた」,「2.他と比較してみた」,「3.コンテストに
チャレンジしてみた」,「4.アプリケーションを作ってみた」のどの課題を実
施したかを書く.実施内容の概要を書く.


\section{提案手法}

課題1の場合は,アルゴリズムの詳細を説明する.また,実装上の工夫を説明す
る.

課題2の場合は,比較手法のアルゴリズムの詳細を説明する.

課題3の場合は,コンテストの目的を1節で述べ,本節では提案アルゴリズムの
詳細を説明する.

課題4の場合は,アプリケーションの目標仕様を1節で述べ,本節では必要とな
る画像処理とその提案アルゴリズムの詳細を述べる.

オープンソースを使用した場合は,その詳細を述べる.また,ソースコードを公
開できる場合は,その入手先を記述する.


\section{実験}

実験の条件を述べる.実験に用いた画像の詳細を述べる.自分で用意した画像の
場合は,カメラの詳細を述べる.計算機の詳細と計算時間を述べる.実験結果の
概要を述べる.


\small
\begin{thebibliography}{10}

\bibitem{gazo2000}
画像太郎, 鈴木一郎:
``画像システム特論レジュメの書き方'', 
日本画像学会誌, vol. 99, no. 4, pp.8-12, 2082.
\bibitem{website}
``画像太郎アルゴリズムのオープンソース'', 
{\tt http:/ /rsj2014.rsj-web.org/}

\end{thebibliography}
\normalsize
\end{document}
